%!TEX root = Main_Assignment2.tex
\documentclass[Main_Assignment2]{subfiles}

\begin{document}
\section{Advantages and disadvantages of the three models}

	\subsection{Problem frames} % (fold)
	\label{sub:problem_frames}

	The context diagrams of problem frames are a good way of figuring in what context your system is going to operate.
	Specifically, it helps you identify what external domain entities affect your system, and which internal domain entities are connected or disjoint.

	It can however at times be difficult to stop oneself from thinking about implementation and unintentionally over specify the context diagram.

	The specific problem frames are a good way of illustrating how requirements are referred to in and constrain the parts of the system context diagram.

	% subsection problem_frames (end)


	\subsection{Z notation}
	The Z notation has certain benefits but also drawbacks.

Benefits:
\vspace{-10pt}
\begin{itemize}
	\item You get a better insight in your system by drawing it.
	\item You might realize the current requirements are not done and can add new requirements to improve the existing ones.
	\item You find global variables you need in the code which you will miss/first discover later
\end{itemize}

Drawbacks:
\vspace{-10pt}
\begin{itemize}
	\item It's harder to read than plain text explaining the requirements.
	\item The benefits gained versus the benefits of other models, using the same time, can be less.
	\item Those who read it must be familiar with the syntax.
\end{itemize}


\end{document}
