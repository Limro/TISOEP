%!TEX root = Main_Assignment3.tex
\documentclass[Main_Assignment3]{subfiles}

\begin{document}
	\section{Configuration Management} % (fold)
		\label{sec:configuration_management}
		For this project it has been chosen to use the Distributed Concurrent Versions System (DCVS) Git.
		Git will be used for maintaining a record of all work on the project in a structured manner.
		A configuration management strategy is needed for the use of Git to have the desired effect.

		The chosen strategy is inspired by \cite{vincentdriessen2010} and is as follows.

		A \texttt{master} development branch will be the center of the source code. Here all finished code will be placed, and testing and metrics will be done on this branch.
		There will not be a \texttt{release} branch, because the nature of the project prohibits incremental release and updates.
		It will not be allowed to merge code that does not compile and run to \texttt{master}.
		Likewise, code with known fault may not be merged into \texttt{master} either.

		Each software component is developed in a separate branch, named intuitively according to the feature description.
		These branches will be merged into \texttt{master} as the features are implemented and unit tested.

		To gauge the progress and estimate code complexity, and hence give an indication on code quality, tests and metrics are run recursively on all new commits, nightly.

	% section configuration_management (end)

\end{document}