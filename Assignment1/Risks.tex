%!TEX root = Main.tex
\documentclass[Main]{subfiles}
\begin{document}

\section{What are the risks and how do you deal with them?}

	\subsection{Risk analysis} % (fold)
	\label{sub:risk_analysis}

	In this section the predicted risks are analyzed.
	The are divided into \emph{Project risks}, \emph{Product risks} and \emph{Business risks}.

	\vspace{-5mm}

		\subsubsection{Project risks} % (fold)
		\label{sub:project_risks}

		The major project risks are related to illness/sick-leave.
		Several employees involved with the development has small children, which are more likely to become sick and require a developer to stay home.

		Also, as development is set to start in September and run through fall, winter and spring, flu-season and the related increase in sick-day frequency will factor into the expected availability of workers.

		Further, as this is purely a software project, there is a risk that employees with a predominant focus on hardware will feel that their skills are under utilized. 
		This could lead to a drop in worker engagement and productivity or it could prompt a worker to seek other challenges elsewhere.
		In any case, an employee choosing to leave in the middle of a project is always a risk.

		If the project involves programming for a new hardware platform, there may be a risk that learning to use a new IDE may prove difficult, and introduce delays.

		% subsection project_risks (end)

		\subsubsection{Product risks} % (fold)
		\label{sub:product_risks}

		A major risk for the product is reliability and stability.
		If the chosen components for maintaining a Real-Time Clock (RTC) are of a too low quality, the product may suffer from difficulties maintaining correct time.

		If the developed software is too inefficient battery life may suffer, introducing a risk for the product to be too in-dependable for users to rely on.

		Likewise, if the software is to unreliable it could miss set alarms, again making is too in-dependable to use.

		% subsection product_risks (end)


		\subsubsection{Business risks} % (fold)
		\label{sub:business_risks}

		The archetypal business risk is a competitor going to market with a similar or better product before the launch of this product.

		In a larger context there is a risk of a greatly diminished market for alarm clocks as more and more people use their cell-phone/smartphone for the same function.

		% subsec tion business_risks (end)

	% subsection risk_analysis (end)

% 	\begin{enumerate}
% 		\item Kids -- should they get sick they need parental care.
% 		\item Sick days for all team members.
% 		\item Team members quiting the job.
% 		\item New tools makes people quit.
% 		\item Software scares the hardware designer away.
% 		\item Competing products are released before this product.
% 		\item Nobody wants to buy a clock -- people has smart phones\dots
% 	\end{enumerate}



	\subsection{Risk assessment and management} % (fold)
	\label{sub:risk_assessment_and_management}

	In this section the risks that are affected by the work of the software development process are discussed.

	The risk of employee having sick days because of own or children's illness, is estimated to be of \emph{high} probability but \emph{low} impact.
	The impact is assessed to be low, because of the loose constraint of schedule.
	To mitigate impact of employee absence, employees are made to work in small groups, so work doesn't stop when one team member is absent.

	The risk of employees with a strong hardware focus feeling that their skills are under utilized is assessed to be of \emph{low} probability, as there still is some focus on hardware interfacing in the project.
	The impact of this risk is assessed to be \emph{high}, as it could lead to the loss of an experienced team member.
	Even though the probability is low, measures to mitigate it is still put in place.
	In this case, all employees are encouraged to attend training to enhance their skill set.

	The risk of a team member choosing to leave is assessed to be of \emph{low} probability, but \emph{high} impact, especially if a team member leaves in the middle of doing a sub-part of the project. 
	This could leave the team with little to no understanding of the work done so far, meaning the work is wasted.
	This is mitigated somewhat by that fact that the employees are made to work in small groups, so no one person is the only one knowing about a specific part of the project.
	Further, employees are made to docuent thier work as they go, so no part is left finished but undocumented and impossible to maintain.





	% subsection risk_assessment_and_management (end)

% To deal with these risk, the company creates a probability and assessment analysis:

% \begin{itemize}
% 	\item[1] Each member with kids are assigned to tasks which are given longer time to finish.
% 	\item[2] Flu season will hit along with common sickness ordinary people tends to catch.
% 	\item[3] Make documentation on the fly.
% 	\item[4] Make time for getting to know the new tools.
% 	\item[5a] The hardware designer must not be set to develop trivial software
% 	\item[5b] Offer the hardware designer classes in software development
% 	\item[6] Don't get delayed.
% 	\item[7] Marketing problem.
% \end{itemize}

\end{document}