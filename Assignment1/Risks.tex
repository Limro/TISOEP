%!TEX root = Main.tex
\documentclass[Main]{subfiles}
\begin{document}

\section{What are the risks and how do you deal with them?}

\subsubsection{Project risks} % (fold)
\label{sub:project_risks}

The major project risks are related to illness/sick-leave.
Several employees involved with the development has small children, which are more likely to become sick and require a developer to stay home.

Also, as development is set to start in September and run through fall, winter and spring, flu-season and the related increase in sick-day frequency will factor into the expected availability of workers.

Further, as this is purely a software project, there is a risk that employees with a predominant focus on hardware will feel that their skills are under utilized. 
This could lead to a drop in worker engagement and productivity or it could prompt a worker to seek other challenges elsewhere.
In any case, an employee choosing to leave in the middle of a project is always a risk.

If the project involves programming for a new hardware platform, there may be a risk that learning to use a new IDE may prove difficult, and introduce delays.

% subsection project_risks (end)

\subsubsection{Product risks} % (fold)
\label{sub:product_risks}

A major risk for the product is reliability and stability.
If the chosen components for maintaining a Real-Time Clock (RTC) are of a too low quality, the product may suffer from difficulties maintaining correct time.

If the developed software is too inefficient battery life may suffer, introducing a risk for the product to be too in-dependable for users to rely on.

Likewise, if the software is to unreliable it could miss set alarms, again making is too in-dependable to use.

% subsection product_risks (end)


\subsubsection{Business risks} % (fold)
\label{sub:business_risks}

The archetypal business risk is a competitor going to market with a similar or better product before the launch of this product.

In a larger context there is a risk of a greatly diminished market for alarm clocks as more and more people use their cell-phone/smartphone for the same function.

% subsec tion business_risks (end)


% 	\begin{enumerate}
% 		\item Kids -- should they get sick they need parental care.
% 		\item Sick days for all team members.
% 		\item Team members quiting the job.
% 		\item New tools makes people quit.
% 		\item Software scares the hardware designer away.
% 		\item Competing products are released before this product.
% 		\item Nobody wants to buy a clock -- people has smart phones\dots
% 	\end{enumerate}

To deal with these risk, the company creates a probability and assessment analysis:

\begin{itemize}
	\item[1] Each member with kids are assigned to tasks which are given longer time to finish.
	\item[2] Flu season will hit along with common sickness ordinary people tends to catch.
	\item[3] Make documentation on the fly.
	\item[4] Make time for getting to know the new tools.
	\item[5a] The hardware designer must not be set to develop trivial software
	\item[5b] Offer the hardware designer classes in software development
	\item[6] Don't get delayed.
	\item[7] Marketing problem.
\end{itemize}

\end{document}