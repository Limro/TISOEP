%!TEX root = Main.tex
\documentclass[Main]{subfiles}

\begin{document}

\section{How do you achieve the appropriate level of quality?}

To maintain an appropriate level of quality there are several steps that can be taken:

\begin{itemize}
	\item Make code reviews
	\item[] By making code reviews the quality of the code will more likely remain high.
	This is mostly necessary because everybody makes mistake.
	This will also prevent later discussion of why code was written as it was, and how it should be written -- thereby making the code in production the same for everyone.

	\item Compare code with specifications
	\item[] When writing code one must keep on track and not code what seems like a good idea or nifty features.
	For each method written, the specifications must state it is needed for the product.
	By doing this the code will remain clean and not evolve to somewhere it should not.


	\item Make sure you follow the same code standard.
	\item[] Select a standard the developers can follow and which ease their overall work. 
	The project will take more than 6 months and the code will be read by almost all team members.
	By having the same standard it will be faster for everyone to read it.

	\item Make sure you test the code, test it properly, and from the beginning.
	\item[] Every line of code is a potential disaster if written wrong.
	Therefor the code must be tested as much as possible. 
	This involves the levels of
	\begin{enumerate}
		\item Drivers to the hardware
		\item Unit test the program
		\item Boundary test
		\item Integration test of the program.
		\item Acceptance test of the program as a whole.
	\end{enumerate}
	Each of these test must be performed multiple times, since small issues can arise when the later are tried.
	The tests starts from the very beginning of the software development and should, if possible, be automated as much as possible.

	\item Write documentation
	\item[] By writing documentation the single developer can get an idea if the code he wrote is good or bad -- hard-to-explain code is often bad written whereas code explained with a pattern is easy to understand.
	\item[] Documentation is also good for understanding the other developers code, since comments in the code not always are enough.
	
\end{itemize}


\end{document}